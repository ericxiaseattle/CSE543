\documentclass{article}

% if you need to pass options to natbib, use, e.g.:
     \PassOptionsToPackage{numbers, compress}{natbib}
% before loading neurips_2021

% ready for submission
%\usepackage{neurips_2021}

% to compile a preprint version, e.g., for submission to arXiv, add add the
% [preprint] option:
     \usepackage[preprint]{neurips_2021}

% to compile a camera-ready version, add the [final] option, e.g.:
%     \usepackage[final]{neurips_2021}

% to avoid loading the natbib package, add option nonatbib:
%    \usepackage[nonatbib]{neurips_2021}

\usepackage[utf8]{inputenc} % allow utf-8 input
\usepackage[T1]{fontenc}    % use 8-bit T1 fonts
\usepackage{hyperref}       % hyperlinks
\usepackage{url}            % simple URL typesetting
\usepackage{booktabs}       % professional-quality tables
\usepackage{amsfonts}       % blackboard math symbols
\usepackage{nicefrac}       % compact symbols for 1/2, etc.
\usepackage{microtype}      % microtypography
\usepackage{xcolor}         % colors

\usepackage{graphicx}    % inserting images
\bibliographystyle{unsrtnat}

\title{Convergence of Optimizers Without Bounded Gradient Assumption}

% The \author macro works with any number of authors. There are two commands
% used to separate the names and addresses of multiple authors: \And and \AND.
%
% Using \And between authors leaves it to LaTeX to determine where to break the
% lines. Using \AND forces a line break at that point. So, if LaTeX puts 3 of 4
% authors names on the first line, and the last on the second line, try using
% \AND instead of \And before the third author name.

\author{%
  Roger Wang, \ Eric Xia \\
  \texttt{rogerwyf@uw.edu, ericxia@uw.edu} \\
}
\usepackage{enumitem}
\usepackage{mathtools}
\usepackage{hyperref}
\usepackage{xcolor}
\hypersetup{
	colorlinks,
	linkcolor={red!50!black},
	citecolor={blue!50!black},
	urlcolor={blue!80!black}
}

\newtheorem{theorem}{Theorem}[section]
\newtheorem{corollary}{Corollary}[theorem]
\newtheorem{lemma}[theorem]{Lemma}

\begin{document}

\maketitle

%\begin{abstract}
%\end{abstract}

\section{Introduction}

The primary goal of most deep learning models is to minimize the loss function.
Optimizers are crucial for the task of updating model weights such that the model will actually converge to a minimum in a computationally efficient manner instead of overshooting or moving away from the minimum. While existing optimizers are intuitively straightforward in convex learning, in non-convex settings they (notably for Adam-type adaptive gradient methods)
often require the assumption on the boundedness of gradients for achieving convergence.

For example, for an objective function  $f(x)$ satisfying
Polyak-Łojasiewicz conditions \cite{POLYAK1963864},  convergence rates of $O(1/T)$ were established for Stochastic Gradient Descent (SGD) under the assumption that $\mathbb{E}[||\nabla f_i(x_k)||^2] \leq C^2$ for all $x_k$ and some $C$, where $f_i$ typically represents the fit on an individual training sample \cite{DBLP:journals/corr/KarimiNS16}.

While convenient, the imposition of assumptions on the boundedness of gradients can be difficult to verify in practical settings, hence in recent years there has been a trend in research efforts towards analyzing existing optimization methods and proposing novel methods without such assumptions. 

In this report, we will survey and analyze results from five latest research papers on the convergence of optimization methods including SGD (\hyperref[section3]{Section 3}), SGD with Momentum (\hyperref[section4]{Section 4}), Root Mean Squared Propagation (RMSProp) (\hyperref[section6]{Section 5}) and Adaptive Moment Estimation (Adam) (\hyperref[section7]{Section 6}), comparing and contrasting their approaches, methodologies, and findings as well as discussing their broader implications for the convergence problem and the general field of optimizers.
\section{General Assumptions}
In a stochastic setting, the optimization problem for a neural network training process can be written as a finite-sum problem:
\[
\min_{x \in \mathbb{R}^d} f(x) = \frac{1}{n}\sum_{j = 0}^{n - 1} f(x, s_j)
\]
where $f(x, s_j)$ represents the loss function contributed by the randomly shuffled sample batch $s_j$.\\
\newline
Let $g_t\coloneqq\nabla f(x_t)$ and $\tilde{g}_t \coloneqq  \nabla f(x_t, s_t)$ be the full gradient and the stochastic gradient with respect to the sampled batch $s_t$ of the objective function $f$ at time or iteration $t$, respectively. Below we list the common assumptions in standard stochastic optimization that are either explicitly stated or implied in the papers covered in this report:
\begin{enumerate}[leftmargin=*]
	\item \textit{\textbf{Lower boundedness}: $f$ is lower-bounded by some function $f^*$.}
	\item \textit{\textbf{Smoothness}: $f$ is $L$-smooth or $\nabla f$ is $L$-Lipschitz continuous for some constant $L$.}
	\item \textit{\textbf{Unbiased gradients}: $\forall t, \mathbb{E}_{s_t}[\tilde{g}_t] = g_t$}.
	\item \textit{\textbf{Independence}: the random samples $s_j$'s are independent.}
	\item \textit{\textbf{Bounded variance}: $Var_{s_t}(\tilde{g}_t) = \mathbb{E}_{s_t}[\tilde{g}_t - g_t] \leq \sigma^2$ for some $\sigma > 0$.}
\end{enumerate}
Throughout this report, we will refer the above as \textit{General Assumptions}. Furthermore, each paper in this report may have additional assumptions which will be included in their corresponding section. It is worth mentioning that none of these assumptions is related to the boundedness to gradient $g_t$, which is often the key assumption utilized by previous theoretical work, and was the first convergence guarantee for SGDM in a multistage setting.
\section{Stochastic Gradient Descent for Structured Nonconvex Functions}
\label{section3}
\subsection{Previous research \& Motivation}
\subsection{Additional Assumptions}
\subsection{Main Result}
\subsection{Key Approaches \& Insights}
\subsection{Discussion}
\section{Stochastic Gradient Descent with Momentum}
\label{section4}
In~\cite{NEURIPS2020_d3f5d4de}, Liu et al. provided a novel convergence analysis for SGDM and Multistage\footnote{Multistage refers to applying a constant stepsize which is then dropped by a constant factor to encourage fine-tuning of training, and the momentum weight is either kept unchanged or gradually increased.} SGDM without bounded gradient assumptions. This work also demonstrated that SGDM has the same convergence bound as SGD for both strongly convex and nonconvex functions without uniformly bounded gradient assumption. 
\subsection{Previous Research \& Motivation}
Prior to this work, there had been some interests in investigating the convergence of SGDM. ~\cite{https://doi.org/10.48550/arxiv.1905.03817} provides a global convergence of SGDM but it assumes uniformly boundedness of gradients of the objection funcion. ~\cite{https://doi.org/10.48550/arxiv.1808.10396} presents a convergence bound of SGDM for general nonconvex functions but does not explain the competitiveness of SGDM compared to SGD. Moreover, the convergence rate of Multistage SGDM had not been established except for the classic SGD case. These issues related to SGDM and Multistage SGDM were addressed in this paper by Liu et al.
\subsection{Main Result}
The main results of this paper are the convergence bounds of SGDM and Multistage SGDM. In SGDM, let $\alpha$ and $\beta$ be learning rate and momentum weight, we have the following result:
\begin{theorem} Assume $f$ satisfies General Assumptions, let $\alpha \leq \min\{\frac{1 - \beta}{L(4 - \beta + \beta^2)}, \frac{1 - \beta}{2\sqrt{2}L\sqrt{\beta + \beta^2}}\}$, then
	\[
	\frac{1}{k}\sum_{i = 1}^{k}\mathbb{E}[\|g_t\|^2] \leq \frac{2 (f(x_1) - f^*}{k\alpha} + (\frac{\beta + 3\beta^2}{2 (1 + \beta)} + 1)L\alpha\sigma^2 = \mathcal{O}(\frac{f(x_1) - f^*}{k\alpha} + L\alpha\sigma^2)
	\]
\end{theorem}
\begin{theorem} Assume $f$ satisfies General Assumptions and is $\mu$-strongly convex, let $\alpha \leq \min\{\frac{1 - \beta}{5L}, \frac{1 - \beta}{L(3 - \beta + 2\beta^2 + \frac{48\sqrt{\beta}}{25}\frac{2L + 18\mu}{L})}\}$, then for all $t \geq t_0:= \lfloor\frac{\log 0.5}{\log \beta}\rfloor$, 
	\[
	\mathbb{E}[f(x_t) - f^*] = \mathcal{O}\bigl(\max\{1 - \alpha\mu, \beta\} + \frac{L}{\mu}\alpha\sigma^2\bigr)
	\]
\end{theorem}
In a multistage setting, let $\alpha_i$, $\beta_i$ and $T_i$ are learning rate (step size), momentum weight and stage length of $i$th stage, respectively, we have the following result:
\begin{theorem} Assume $f$ satisfies General Assumptions, restrict the parameters in each stage of Multistage SGDM so that 
\begin{equation}
\begin{split}
\frac{\alpha_i\beta_i}{1 - \beta_i} &\equiv A_1 \ \ \ \ i = 1,...,n\\
\alpha_iT_i &\equiv A_2 \ \ \ \ i = 1,...,n\\
0 \leq \beta_1 &\leq \beta_2 \leq ... \leq \beta_n \leq 1
\end{split}
\end{equation}
and $A_1, A_2$ are properly chosen constants. Let $A_1 = \frac{1}{24\sqrt{2}L}$ and $A_2$ be large enough so that $\beta_i^{2T_i} \leq \frac{1}{2}$ for $i = 1,...,n$. In addition, let
\[
\frac{1 - \beta_1}{\beta_1} \leq 12 \frac{1 - \beta_n}{\sqrt{\beta_n + \beta_n^2}}
\] then we have
\[
\begin{split}
\frac{1}{n}\sum_{l = 1}^{n}\frac{1}{T_l}\sum_{i = T_1 + ... + T_{l - 1} + 1}^{T_1 + .. + T_l}\mathbb{E}[\|g_t\|^2] &\leq \frac{2(f(x_1) - f^*)}{nA_2} + \frac{1}{n}\sum_{i = 1}^n\Bigl(24\beta^2_l\frac{\beta_1}{\sqrt{\beta_n + \beta_n^2}L + 3L}\Bigr)\alpha_l\sigma^2\\
&= \mathcal{O}\Bigl(\frac{2(f(x_1) - f^*)}{nA_2} + \frac{1}{n}\sum_{i = 1}^nL\alpha_l\sigma^2\Bigr)
\end{split}
\]
\end{theorem}
\subsection{Key Approaches \& Insights}
Recall that the core of SGDM algorithm is the following updating rule:
\[
v_t = \beta v_{t - 1} + (1 - \beta)\tilde{g}_t \ \ \ \
x_{t + 1} = x_t - \alpha v_t
\]
Therefore, assume $v_0 = 0$, then $v_t$ can be expressed as 
\begin{equation}
v_t = (1 - \beta)\sum_{i = 1}^{t}\beta^{k - i}\tilde{g}_i
\end{equation}
One key observation on the role of $\beta$ in equation (2) from the paper is that $v_t$ enjoys a reduced variance of $(1 - \beta)\sigma^2$ while having a controllable deviation from the full gradient $g_t$ in expectation since $v_t$ is a moving average of the past stochastic gradients with lower weights on the older ones, thus it makes sense to look at the deterministic version of $v_t$ (replacing $\tilde{g}_i$ with $g_i$) and its deviation from the ideal descent direction $g_t$, which could be unbounded without further assumptions.

While previous work assumed the boundedness of $g_t$ to circumvent above difficulty, this work constructed a novel Lyapunov function to handle this deviation:
\begin{equation}
L_t = (f(z_t) - f^*) + \sum_{i = 1}^{k - 1}c_i\|x^{k + 1 - i} - x^{k - i}\|^2
\end{equation}
where $z_t = \begin{cases}
	x_t & t = 1 \\
	\frac{1}{1 - \beta}x_t - \frac{\beta}{1 - \beta}x_{t - 1} & t \geq 2
\end{cases}$\\
\ \\
The authors then argued that by carefully defining $\{c_i\}_i^\infty$ such that it is a positive sequence in a diminishing fashion, $L_t$ is indeed a Lyapunov function, thus one can show that $\mathbb{E}[L_{t + 1} - L_t] \leq -R_1E[\|g_t\|^2] + R_2$ for some positive constants $R_1 \geq \frac{\alpha}{2}$ and $R_2 = \mathcal{O}(L\alpha\sigma^2)$. By telescoping this inequality, the convergence of SGDM in Theorem 4.1 is then obtained, and similar techniques were utilized to derive the results in Theorem 4.2 under a strongly convex setting and in Theorem 4.3 for Multistage SGDM.
\subsection{Discussion}
Theorem 4.1 and Theorem 4.2 show that under both nonconvex and strongly convex settings, with a proper learning rate $\alpha$, SGDM can achieve the same convergence bound as the classical convergence bound of SGD (as shown in previous work, e.g., Theorem 4.5 and 4.8 in~\cite{https://doi.org/10.48550/arxiv.1606.04838}). This result only depends on General Assumptions, and the radius of the stationary distribution is smaller than the previous $\mathcal{O}(\frac{\alpha\sigma^2}{1 - \beta})$ result from~\cite{https://doi.org/10.48550/arxiv.1905.03817} that relies on the additional assumption of uniformly bounded gradients. It is also worth mentioning that the use of Lyapunov function is a novel approach for convergence analysis of optimization algorithms and provides some new insights throughout this paper. 

For Multistage SGDM, Theorem 4.3 is the first theoretical result that guarantees its convergence. Moreover, it was demonstrated from the convergence bound that large learning rates are allowed in the first a few stages to accelerate the initial convergence, and smaller learning rates can refine the radius of the stationary distribution in the later stages, which is an advantage of stagewise training compared to plain SGDM.

However, the convergence analysis in this paper does have some weaknesses and limitations: First of all, although it is theoretically shown that SGDM is "at least as fast as" SGD, this paper did not explore the advantages of SGDM compared to SGD in detail. In addition, Theorem 4.2 assumes a lower-bound of timestamp/iteration for the result to be valid, and this lower-bound could be a problem for some choices of $\beta$ (e.g, when $\beta = 0.995$, $t_0 = 138$). Finally, equation (1) from Theorem 4.3 puts a strong restriction on the choice of learning rates and momentum weights at all stages, which makes this stage-wise training setup impractical.
\section{RMSProp}
\label{section6}
\subsection{Previous research \& Motivation}
\subsection{Additional Assumptions}
\subsection{Main Result}
\subsection{Key Approaches \& Insights}
\subsection{Discussion}
\section{Adam}
\label{section7}
\subsection{Previous research \& Motivation}
\subsection{Additional Assumptions}
\subsection{Main Result}
\subsection{Key Approaches \& Insights}
\subsection{Discussion}
\section{Addition Dicussion \& Comments}
\section{Future Research}
\newpage
\bibliography{final}

%\appendix
%\section{Appendix}
%
%Optionally include extra information (complete proofs, additional experiments and plots) in the appendix.
%This section will often be part of the supplemental material.

\end{document}
